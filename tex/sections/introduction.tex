\section{はじめに: SYK模型とは}
Sachdev--Ye--Kitaev(SYK)模型とは、KitaevがAdS/CFT対応の簡単な模型として提唱した
ものであり、そのハミルトニアンは次のように与えられる:
\begin{align}
  H = \frac{1}{4!}\sum_{i,j,k,l = 1}^{N} J_{ijkl}\psi_i\psi_j\psi_k\psi_l.
  \label{eq:hamiltonian}
\end{align}
ここで$J_{ijkl}$は乱数で与えられる反対称テンソルであり、その分布は
\begin{align}
  P(J_{ijkl}) = \sqrt{\frac{N^3}{12\pi J^2}}
                \exp\left(-\frac{N^3}{12J^2}J_{ijkl}^{\, 2}\right).
	\label{eq:gaussianProbability}
\end{align}
に従う。
SYK模型において次元を持つパラメータのひとつは\eqref{eq:gaussianProbability}式の$J$であり、
$[J] = 1$である。\eqref{eq:hamiltonian}式の大きさはこの$J$によって決まる。

SYK模型を調べるモチベーションは複数あり、大別して以下の3つに分けることができる:\\

\textbf{強結合領域で可解な模型である.}
ラージN極限を取るとリーディングオーダーでのファインマンダイアグラムが単純なものとなり、
その和を取ることによって強結合領域での相関関数が計算可能である。\\

\textbf{最大にカオスである.}
カオスはリャプノフ指数によって計られ、その最大値はアインシュタイン重力における
ブラックホールが持ち、$2\pi / \beta$となる\cite{shenker}。
強結合領域におけるSYK模型もこの最大値を満たす事が知られている\cite{maldacena}。\\

\textbf{共形不変性が現れる.}
低エネルギーでは2点関数が共形不変性を持つ。\\

大抵の場合、理論を解析する際には摂動論が適用できる範囲でしか計算する事ができないが、
SYK模型では一つ目の性質により強結合領域でも計算する事が可能であり、大きな
モチベーションの一つとなっている。
一つ目と二つ目を組み合わせると非常に興味深い。
古典論では「可解である」という事と可積分系である事は等しく、従ってカオスとは相容れない\cite{polchinski}。
これが量子論では必ずしもそうではない事をSYK模型は示している。
また二つ目と三つ目の性質により、何らかの形におけるアインシュタイン重力理論がAdS/CFT対応
での双対理論として期待される。

\ref{sec:twopointfunc}節ではSYK模型のラージ$N$極限における構造を調べる。
SYK模型はある確率分布に従う乱数$J_{ijkl}$を持つ。
\ref{sec:twopointfunc}節の諸々の結果は$J_{ijkl}$に対して平均操作を施したものである。
この結果、2つのbi-local場$G(t_1, t_2)$と$\Sigma(t_1, t_2)$が現れる。
分配関数においてフェルミオンを積分して取り払うと、これらのbi-local場についての作用を得る事ができ、
$G$や$\Sigma$について変分を取ればそれらに関する運動方程式(シュウィンガー・ダイソン方程式)を得る。
$G$の古典解はフェルミオンの2点関数に等しい。
シュウィンガー・ダイソン方程式は一般のエネルギースケールにおける解析解は知られておらず、
数値的な計算がメインである。
また低エネルギー極限では共形対称性を持ち、解析的な解の具体型も存在する。
解析解が知られているケースはこの場合以外にも、
相互作用するフェルミオンの数$q$についてラージ極限を取った場合や、
$q=2$とした場合もある。
特にラージ$q$について得た$G$を用いる事で自由エネルギーやエントロピーを$1/q$で展開した表式で
得る事ができる。
これらの諸々の熱力学的量は後の量子カオスにおける性質を述べる上で重要な役割を持つ。

\pagebreak
