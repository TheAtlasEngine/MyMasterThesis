\section{はじめに: SYK模型とは}
Sachdev--Ye--Kitaev(SYK)模型とは, KitaevがAdS/CFT対応の簡単な模型として提唱した
ものであり, そのハミルトニアンは次のように与えられる:
\begin{align}
  H = \frac{1}{4!}\sum_{i,j,k,l = 1}^{N} J_{ijkl}\psi_i\psi_j\psi_k\psi_l.
  \label{eq:hamiltonian}
\end{align}
ここで$\psi_i = \psi_i(t)$はマヨラナフェルミオンである. 
また$J_{ijkl}$は乱数で与えられる時間に依らない反対称テンソルであり, その分布は
\begin{align}
  P(J_{ijkl}) = \sqrt{\frac{N^3}{12\pi J^2}}
                \exp\left(-\frac{N^3}{12J^2}J_{ijkl}^{\, 2}\right)
	\label{eq:gaussianProbability}
\end{align}
に従う. 
このようなパラメータを持つ理論は quenched disorder を持つといい, 
この$J_{ijkl}$で取った平均を disorder averageと呼ぶ. 
SYK模型において次元を持つパラメータのひとつは\eqref{eq:gaussianProbability}式の$J$であり, 
その質量次元は1である. \eqref{eq:hamiltonian}式の大きさはこの$J$によって決まる. 
SYK模型のヒルベルト空間は
\begin{align}
	L = 2^{N/2}
\end{align}
という次元を持つ. 

SYK模型を調べるモチベーションは複数あり, 大別して以下の3つに分けることができる:\\

\textbf{強結合領域で可解な模型である.}
ラージN極限を取るとリーディングオーダーでのファインマンダイアグラムが単純なものとなり, 
その和を取ることによって強結合領域での相関関数が計算可能である. \\

\textbf{量子カオス系である.}
カオスはリャプノフ指数と呼ばれる量によって計られ, 
その最大値は$2\pi / \beta$となる事が証明されている \cite{shenker}. 
強結合領域におけるSYK模型はこの最大値を満たし\cite{maldacena}, 
SYK模型は量子カオス系となっている. \\

\textbf{共形不変性が現れる.}
低エネルギーでは2点関数が共形不変性を持つ. \\

大抵の場合, 理論を解析する際には摂動論が適用できる範囲でしか計算する事ができないが, 
SYK模型では一つ目の性質により強結合領域でも計算する事が可能であり, 大きな
モチベーションの一つとなっている. 
1つ目と2つ目を組み合わせると非常に興味深い. 
古典論では「可解である」という事と可積分系である事は等しく, 従ってカオスとは相容れない\cite{polchinski}. 
これが量子論では必ずしもそうではない事をSYK模型は示している. 
また3つ目はSYK模型がAdS/CFT対応を満たす事を示唆している。
実際, 低エネルギーではシュワルツ理論によって記述される作用がSYK模型への主要な寄与となり, 
またシュワルツ理論はJackiw--Teitelboimディラトン重力理論と呼ばれるある種のアインシュタイン重力と
等価である事が知られている.

さらに, SYK模型が量子カオス系であるという性質は重力双対が存在する事のもう1つの動機付けとなる.
カオスの度合いを測るリャプノフ指数は上述したように$\beta / 2\pi$という上限値を持つ.
リャプノフ指数は重力理論でも計算できるものであり, その値は上限値$\beta / 2\pi$に一致する事が示されている
\cite{shenker}.
従ってSYK模型と重力理論は``同程度にカオス"であり, かつ最も強いカオスを持つ
\footnote{ただしリャプノフ指数を持つならば量子カオス系であるとは限らないという議論が\cite{eiki}
によってなされている. 逆は成り立つ.}。


本修論のメインの目標はSYK模型の満たすAdS/CFT対応を通してブラックホールの振る舞いを調べる事である. 
特にブラックホールの持つ量子カオス的性質を探る. 
これは究極的にはブラックホールの情報喪失問題につながるものであり, 
相対性理論の時空の滑らかさと量子論のヒルベルト空間の離散性の間の深い関係に根付く\cite{stanford_chaos}. 
以下では本修論の構造を説明する. 

第\ref{sec:twopointfunc}節ではSYK模型のラージ$N$極限における構造を調べる. 
SYK模型はある確率分布に従う乱数$J_{ijkl}$を持ち, 
第\ref{sec:twopointfunc}節の諸々の結果は$J_{ijkl}$に対して平均操作を施したものである. 
特に分配関数についてこの平均操作を行うと
2つの双局所場(bi-local field)$G(t_1, t_2)$と$\Sigma(t_1, t_2)$が現れ, 
フェルミオンを積分して取り払うとこれらの双局所場についての作用を得る事ができ, 
$G$や$\Sigma$について変分を取ればそれらに関する運動方程式(シュウィンガー・ダイソン方程式)を得る. 
$G$の古典解はフェルミオンの2点関数に等しい. 
SYK模型のシュウィンガー・ダイソン方程式は一般のエネルギースケールにおける解析解は知られておらず, 
数値的な計算がメインである. 
また低エネルギー極限では共形対称性を持ち, 解析的な解の具体型も存在する. 
解析解が知られているケースはこの場合以外にも, 
相互作用するフェルミオンの数$q$についてラージ極限を取った場合や, 
$q=2$とした場合もある(ただし本論文では$q=2$のケースは重要ではないので扱わない). 
特にラージ$q$について得た$G$を用いる事で自由エネルギーやエントロピーを$1/q$で展開した表式で
得る事ができる. 
これらの諸々の熱力学的量は後の量子カオスにおける性質を述べる上で重要な役割を持つ. 

第\ref{sec:fourpointfunc}節では4点関数の解析解について述べる. 
特にラージ$N$におけるリーディングオーダー$\mathcal{F}$を調べるのだが, ダイアグラムは書き下せても
一般的な解析解は知られていない. 
$\mathcal{F}$に寄与するダイアグラムはラダーダイアグラムと呼ばれるものであり, $\mathcal{F}$は
その総和で与えられる. 
ラダーダイアグラムは図\ref{fig:ladderdiagram}のように, 両側にレールがあり, 
それに接するように内側に輪が存在する. 
$n$個, および$n+1$個の輪を持つダイアグラムの間には積分核$K$で与えられる漸化式が存在し, 
ダイアグラムの総和は$K$の幾何級数で与えられる. 
従ってリーディングオーダーは$\mathcal{F} = \frac{1}{1-K}\mathcal{F}_0$で与えられる. 
ここで$\mathcal{F}_0$は輪を持たないダイアグラムである. 
概念的には簡単であるが, $K$の作用する関数空間についてある程度理解する必要があり, 
実際の計算はとても複雑である. 
そこで2点関数では共形極限における解析解$G_c$が調べられている事を思い出し, 
4点関数でもそれを用いる事にする. 
基本的に4点関数のリーディングオーダー$\mathcal{F}$は2点関数で構成されるので, 
原理的には$G_c$を用いて解析は可能であり, 実際に解を導く. 
ただしそれでも, そこまでの道のりがかなり長い. 
一次元における共形対称性は$SL(2, \mathbb{R})$であり, これを$K$の対角化において活用する. 
これによって一応の計算結果が示されるが, 結果の表式に存在する級数は
低エネルギー極限で計算した事に起因する技術上の発散項を含む. 
この発散項は, $K = 1$となるような$K$の固有関数の存在によってもたらされる. 
この発散を処理するためには, リパラメトリゼーション不変性の成り立つ低エネルギー極限から
少し高エネルギー側にずれる必要がある. 
すなわち, 有限な解ではリパラメトリゼーション不変性は自発的にも, また陽にも破れる. 
発散項はこれによって有限になり, 共形対称性を持たない. 

第\ref{sec:effective_theory}節ではSYK模型の有効理論を扱う. 
低エネルギー極限から少し離れたエネルギーで議論し, 共形対称性の自発的破れによる
ゴールドストーンボソンはシュワルツ理論に従う事を見る. 
既に述べたように、このシュワルツ理論は一般にJackiw--Teitelbiom重力理論と等価である事が知られている. 
従ってAdS/CFT対応が期待されるが, シュワルツ理論がSYK模型への主要な寄与となるようなエネルギースケールでは
共形対称性が破れているため, 正しくはNear AdS/Near CFT対応と呼ぶべきものとなっている. 

第\ref{sec:thermodynamics}節ではSYK模型に登場する熱力学的な量を計算する. 
特に後に重要となるのはエントロピーと比熱であり, 量子カオスの分野に
おいて重要なスペクトラル形状因子という量のrampの傾きやplateauの高さなどの振る舞いを司る. 

第\ref{sec:syk_as_quantum_chaos}節ではSYK模型の量子カオスとの関連を議論する. 
量子カオスの分野の基本的な仮定として, 量子カオス系のスペクトラル統計はランダム行列理論と呼ばれる数学で
記述されるというものがある. 
スペクトラル統計の重要な量の1つにスペクトラル形状因子というものがあり, これがランダム行列理論やSYK模型
でよく調べられている. 
スペクトラル形状因子は系のlate-time (t \gg 1の時間領域)の振る舞いを探るのに適した量であり, 
量子系であるならばrampやplateauという特有の構造を持つ. 
SYK模型のスペクトラル形状因子もrampやplateauと呼ばれる領域を持ち, これが
ある近似のもとでランダム行列理論で記述可能である事を見る. 
ここまではスペクトラル形状因子をラージ$N$で直接計算したものであるが, 次に
フェルミオンの2点関数や自己エネルギーによる記述を行う. 
これによって第\ref{sec:gravity}節で議論するSYK模型の重力双対である
Jackiw--Teitelboim重力理論におけるスペクトラル形状因子に相当する量の構造と
SYK模型でのそれとの対応が見れるようになる. 

\pagebreak
