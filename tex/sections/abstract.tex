\begin{abstract}
本修士論文は量子カオス系としてのSachdev-Ye-Kitaev (SYK)模型
およびその重力双対のブラックホールを研究したものである。
ブラックホールの時間的な振る舞いは具体的にわかっていない事が多く, 
特にブラックホールの情報喪失問題は, 
一般相対性理論の時空の滑らかさと量子論のヒルベルト空間の離散性の関係に基づく非常に根の深い問題であり, 
これを理解するためには量子重力の持つ性質の明確化が必要である. 
しかし量子重力は現在のところ厳密な定義が存在しないものであり, その研究は困難を極める. 

量子重力の研究の重要な鍵とされているものにホログラフィック原理というものがある. 
これは共形対称性を有する場の理論と重力理論との間には対応関係が存在するというものであり, 
これを用いれば, 例えば弦理論に登場する諸々の量を超対称性非可換ゲージ理論の言葉に置き換えるという
ある種の翻訳作業が可能である. 
従って十分に解明されていない量子重力の性質を, よく知られている場の理論の性質として理解できる. 

しかしながらホログラフィック原理を満たすとされている理論はその解析が困難であるものがほとんどであり, 
ホログラフィック原理そのものの理解も停滞していた. 
その中でSYK模型はホログラフィック原理に従う模型の中でも比較的解析の簡単なものとされており, 
現在盛んに研究が行われている. 

SYK模型は低エネルギー極限で共形対称性を持ち, その重力双対はJackiw-Teitelboim (JT)重力理論である. 
またSYK模型は量子カオス系であり, 
量子カオス的性質を調べるための道具の1つであるスペクトラル形状因子の構造を具体的に探ると, 
slope, rampやplateauという3つの領域を持つ事がわかる. 
量子カオスを記述する数学はランダム行列理論であるという期待が存在し, 
実際にSYK模型のスペクトラル形状因子もランダム行列理論で記述される事を見る. 

スペクトラル形状因子に相当する量はJT重力理論にも存在し, 
この重力理論とSYK模型の間には明確な対応関係がある事も見る. 
スペクトラル形状因子の解析は理論の分配関数の中身の経路積分を含むため, 
重力側の解析は時空構造そのものの探求でもある. 

しかしplateau領域の解析は現時点では行われておらず, これについて少しばかりの議論を最後に行う. 
\end{abstract}