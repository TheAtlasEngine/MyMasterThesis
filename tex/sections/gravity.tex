\section{量子カオスとしてのブラックホール\label{sec:gravity}}
この節では主にSYK模型の重力双対であるJackiw--Teitelboim(JT)ブラックホールの量子カオスの性質を述べる
\footnote{この節の内容は\cite{polchinski_chaos}及び\cite{stanford_chaos}による。}。
特に\eqref{eq:spectral_form_factor}式で定義されるスペクトラル形状因子の
slopeとrampを与えるような時空の計量はどのようなものかという事を議論する。
ここで注意するべきは、JTディラトン重力はSYK模型のようなdisorderを持つ理論ではないため、
$\average{ZZ^*}$のような期待値計算ではなく$ZZ^*$そのものの計算となるという事である
\footnote{SYK模型においても期待値を外して$ZZ^*$を計算する事はもちろん可能である。
この時にrampやplateauを数値解析によりプロットすると大きい揺らぎを目視する事ができる。
一方でこの節で計算するJT重力の$ZZ^*$には、disorder averageを取っていないにも関わらず、
揺らぎが存在しない。この点は今後の課題の1つである。詳しくは第\ref{sec:conclusion}節を参照。}。

第\ref{sec:fourpointfunc}節で論じたように、SYK模型は低エネルギーでシュワルツ理論となるため、
前節で論じた鞍点もシュワルツ理論によって記述できる。
一般にシュワルツ理論はJTディラトン重力理論と等価である事が知られており、従って前節の$G$と$\Sigma$の配位は
AdS/CFT対応におけるバルク側のJTディラトン重力として理解できる。
ただし$G$と$\Sigma$の配位というのは重力側では計量の配位として翻訳されるべきものである。
従ってここでの問題はSYK模型で議論した鞍点は重力側ではどのような時空の幾何学に対応するかという事である。

前節と同様にここでも2つのレプリカが現れ、それぞれL系とR系と名付ける事にすると、
前節の$G$と$\Sigma$の配位に全く対応した解が現れる。
即ち、L系とR系に相関がないような解がslopeを与え、また両レプリカ系に相関が存在し、
かつ各々が独立に持っていた時間並進対称性が自発的に破れるような解はrampとして寄与する。
ちなみに一般相対性理論で計量のアイソメトリーを意味するKillingベクトル場と対応して、
JTディラトン重力理論の計量が時間方向の並進でアイソメトリーを持つ事から、
この時間座標をKilling時間と呼ぶ事にする。

L系とR系はここではdiskのトポロジーを持つブラックホールに対応し、特にrampを与えるような解は
この2つのブラックホールをつなぐようなワームホールと呼ばれるものとなる。
以下では上述した事の詳細を述べる。

\subsection{JTディラトン重力}
まず最初にJTディラトン重力について簡単に紹介する。
作用は次式で与えられる:
\begin{align}
	S_{JT} = -\frac{\phi_0}{2}\left[\int \sqrt{g}R + 2\int_{bdy}\sqrt{h}K \right]
			-\frac{1}{2}\left[
				\int \sqrt{g}\phi(R+2) + 2\phi_b\int_{bdy}\sqrt{h}K
			\right].
\end{align}
この理論は高次元のAdS空間を2次元にコンパクト化する事で得られる。
ディラトン$\phi$はその際のコンパクト化半径に対応する。
$\phi$について変分する事で得られる運動方程式はリッチスカラーに関する方程式であり、$R = -2$となる。
これにより2次元においては局所的な時空の幾何が$\mathrm{AdS}_2$として固定される。


\pagebreak