\section{係数$b$の計算 \label{app:b}}
	\renewcommand{\theequation}{B.\arabic{equation}}
	\setcounter{equation}{0}
	
	この付録では\eqref{eq:conformal_ansatz}式の係数$b$の計算方法について述べる.
	まず\eqref{eq:conformalSD}式の2つの式を一つにまとめると
	\begin{align}
		J^2\int ds\ G(s - t_1)G(s - t_2)^{q - 1} = -\delta(t_2 - t_1)
		\label{eq:intOfGG}
	\end{align}
	となる.この積分を実行するには、
	\begin{align}
	\frac{\sgn(t)}{|t|^{2\Delta}} = 
		\int \frac{d\omega}{2\pi}\exp(-i\omega t)\ 
		i\, 2^{1 - 2\Delta}\sqrt{\pi}\frac{\Gamma(1 - \Delta)}{\Gamma(1/2 + \Delta)}
		|\omega|^{2\Delta - 1}\sgn(\omega)
	\end{align}
	を使用すると便利である.
	少しの計算のあとに、
	\begin{align}
	-\delta(t) = &-J^2b^q\pi
		\frac{\Gamma(1 - \Delta)\Gamma(\Delta)}{\Gamma(1/2 + \Delta)\Gamma(3/2 - \Delta)}
		\nonumber\\
		&\times \int ds\ 
		\int \frac{d\omega}{2\pi}\ \exp(-i\omega(t - s))\ |\omega|^{2\Delta - 1}\sgn(\omega)
		\int \frac{d\Omega}{2\pi}\ \exp(-i\Omega s)\ |\Omega|^{1 - 2\Delta}\sgn(\Omega)
	\end{align}
	という式にたどり着く。
	積分の部分は最初に$s$について実行すると$\delta(\omega - \Omega)$が現れ、その後
	$\omega$および$\Omega$で積分すると$\delta(t)$が現れる。
	またガンマ関数の部分については
	\begin{align}
	\Gamma(z)\Gamma(1 - z) = \frac{\pi}{\sin(\pi z)}
	\hspace{30pt}
	{\rm for}\ {}^{\forall}z \notin \mathbb{Z}
	\end{align}
	という性質と$\Gamma(1 + z)\Gamma(z) = z\Gamma(z)$を用いることで
	\begin{align}
	\frac{\Gamma(1 - \Delta)\Gamma(\Delta)}{\Gamma(1/2 + \Delta)\Gamma(3/2 - \Delta)}
	= \frac{1}{(1/2 - \Delta) \tan(\pi\Delta)}
	\end{align}
	となる. 以上より、係数$b$は
	\begin{align}
		J^2b^q\pi = \left(\frac{1}{2} - \Delta\right)\tan(\pi\Delta)
	\end{align}
	という式により決定できる。
	
	\pagebreak